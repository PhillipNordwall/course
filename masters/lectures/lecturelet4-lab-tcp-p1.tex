% Included from both -slides and -handout versions.

\mode<presentation>
{
  \usetheme{default}
  \useoutertheme{infolines}
}

\usepackage[english]{babel}
\usepackage[latin1]{inputenc}
\usepackage{graphicx}
\usepackage{times}
\usepackage[T1]{fontenc}
\usepackage{fancyvrb}
\usepackage{hyperref}
\usepackage{listings}
\begin{document}
\lstset{language=C, escapeinside={(*@}{@*)}, numbers=left,
  basicstyle=\tiny, showspaces=false, showtabs=false}

\def\Tiny{\fontsize{4pt}{4pt} \selectfont}

\title{L41: Lab 4 - The TCP State Machine}
\author{Dr Robert N. M. Watson}
\date{29 January 2016}

\begin{frame}
  \titlepage
\end{frame}

\section{Introduction}

\begin{frame}
  \frametitle{L41: Lab 4 - The TCP State Machine}

  \begin{itemize}
    \item The TCP state machine
    \item Setting the MTU, IPFW, and DUMMYNET
    \item TCP mode for the IPC benchmark
    \item DTrace probes of interest
    \item Experimental and exploratory questiond
  \end{itemize}
\end{frame}

\begin{frame}[fragile]
  \frametitle{Lect 6: The Transmission Control Protocol (TCP)}

  \begin{columns}[T]
    \column{0.45\textwidth}

  \begin{Tiny}
    \begin{verbatim}
September 1981                             Transmission Control Protocol
                                                Functional Specification

                              +---------+ ---------\      active OPEN  
                              |  CLOSED |            \    -----------  
                              +---------+<---------\   \   create TCB  
                                |     ^              \   \  snd SYN    
                   passive OPEN |     |   CLOSE        \   \           
                   ------------ |     | ----------       \   \         
                    create TCB  |     | delete TCB         \   \       
                                V     |                      \   \     
                              +---------+            CLOSE    |    \   
                              |  LISTEN |          ---------- |     |  
                              +---------+          delete TCB |     |  
                   rcv SYN      |     |     SEND              |     |  
                  -----------   |     |    -------            |     V  
 +---------+      snd SYN,ACK  /       \   snd SYN          +---------+
 |         |<-----------------           ------------------>|         |
 |   SYN   |                    rcv SYN                     |   SYN   |
 |   RCVD  |<-----------------------------------------------|   SENT  |
 |         |                    snd ACK                     |         |
 |         |------------------           -------------------|         |
 +---------+   rcv ACK of SYN  \       /  rcv SYN,ACK       +---------+
   |           --------------   |     |   -----------                  
   |                  x         |     |     snd ACK                    
   |                            V     V                                
   |  CLOSE                   +---------+                              
   | -------                  |  ESTAB  |                              
   | snd FIN                  +---------+                              
   |                   CLOSE    |     |    rcv FIN                     
   V                  -------   |     |    -------                     
 +---------+          snd FIN  /       \   snd ACK          +---------+
 |  FIN    |<-----------------           ------------------>|  CLOSE  |
 | WAIT-1  |------------------                              |   WAIT  |
 +---------+          rcv FIN  \                            +---------+
   | rcv ACK of FIN   -------   |                            CLOSE  |  
   | --------------   snd ACK   |                           ------- |  
   V        x                   V                           snd FIN V  
 +---------+                  +---------+                   +---------+
 |FINWAIT-2|                  | CLOSING |                   | LAST-ACK|
 +---------+                  +---------+                   +---------+
   |                rcv ACK of FIN |                 rcv ACK of FIN |  
   |  rcv FIN       -------------- |    Timeout=2MSL -------------- |  
   |  -------              x       V    ------------        x       V  
    \ snd ACK                 +---------+delete TCB         +---------+
     ------------------------>|TIME WAIT|------------------>| CLOSED  |
                              +---------+                   +---------+

                      TCP Connection State Diagram
                               Figure 6.
\end{verbatim}
  \end{Tiny}

    \column{0.45\textwidth}

    \bigskip
    \pause

    \begin{itemize}
      \item V. Cerf, K. Dalal, and C. Sunshine, \textit{Transmission Control
	Protocol (version 1)}, INWG General Note \#72, December 1974.
      \item In practice: Jon Postel, Ed, \textit{Transmission Control
	Protocol: Protocol Specification}, RFC 793, September, 1981.
    \end{itemize}
  \end{columns}

\end{frame}

\begin{frame}
  \frametitle{Lect 6: TCP goals and properties}

  \begin{columns}[T]
    \column{0.38\textwidth}
      \smallskip
      \begin{center}
	\includegraphics[width=1.15\textwidth]{../../figures/tcp-timeline.pdf}
      \end{center}

    \column{0.52\textwidth}

      \pause

      \begin{itemize}
	\item Network may delay, (reorder), drop, corrupt packets

	\pause

	\item TCP: Reliable, ordered, stream transport protocol over IP

	%\item Connections identified by unique IP-port 4-tuple

	\pause

	\item Three-way handshake: SYN~/ SYN-ACK~/ ACK (mostly!)

	\pause

	\item Sequence numbers ACK'd; data retransmitted on loss

	\pause

	\item Round-Trip Time (RTT) measured to time out loss

	\pause

        \item Flow control via advertised window size in ACKs

	\pause

	\item Congestion control (`fairness') via packet loss and ECN

	%\pause

	%\item Complex teardown permits `half-close' and port reuse
      \end{itemize}

  \end{columns}
\end{frame}

\begin{frame}[fragile]
  \frametitle{Loopback interface, IPFW, and DUMMYNET}

  \begin{itemize}
    \item Network-stack features to configure \textbf{once per boot}

    \pause

    \item Loopback interface
    \begin{itemize}
      \item Simulated local network interface: packets ``loop back''
      \item Interface name \texttt{lo0}
      \item Assigned IPv4 address 127.0.0.1
    \end{itemize}

    \pause

    \item IPFW - IP firewall by Rizzo, et al.
    \begin{itemize}
      \item Numbered rules classify packets and perform actions
      \item Actions include accept, reject, inject into DUMMYNET ...
      \item We will match lab flows using the TCP port number 10,141
    \end{itemize}

    \medskip
    \pause

    \item Configure (and reconfigure) \textbf{for each experiment}
    \item DUMMYNET - link simulation tool by Rizzo, et al.
    \begin{itemize}
      \item Widely used in network research
      \item Impose simulated network conditions -- delay, bandwidth, loss, ...
    \end{itemize}
  \end{itemize}
\end{frame}

\begin{frame}[fragile]
  \frametitle{TCP in the IPC benchmark}

  \begin{tiny}
    \begin{verbatim}
root@beaglebone:/data/ipc # ./ipc-static 
ipc-static [-Bqsv] [-b buffersize] [-i pipe|local|tcp] [-p tcp_port]
        [-P l1d|l1i|l2|mem|tlb|axi] [-t totalsize] mode

Modes (pick one - default 1thread):
    1thread                IPC within a single thread
    2thread                IPC between two threads in one process
    2proc                  IPC between two threads in two different processes

Optional flags:
    -B                     Run in bare mode: no preparatory activities
    -i pipe|local|tcp      Select pipe, local sockets, or TCP (default: pipe)
    -p tcp_port            Set TCP port number (default: 10141)
    -P l1d|l1i|l2|mem|tlb|axi  Enable hardware performance counters
    -q                     Just run the benchmark, don't print stuff out
    -s                     Set send/receive socket-buffer sizes to buffersize
    -v                     Provide a verbose benchmark description
    -b buffersize          Specify a buffer size (default: 131072)
    -t totalsize           Specify total I/O size (default: 16777216)
\end{verbatim}
  \end{tiny}

  \begin{itemize}
    \item \texttt{tcp} IPC type
    \item \texttt{-p} argument to set the port number
  \end{itemize}
\end{frame}

\begin{frame}
  \frametitle{DTrace probes}

  Described in more detail in the lab assignment:

  \medskip

  \begin{description}
    \item[fbt::syncache\_add:entry] TCP segment installs new SYN-cache entry
    \item[fbt::syncache\_expand:entry] TCP segment converts SYN-cache entry to
      full connection
    \item[fbt::tcp\_do\_segment:entry] TCP segment received post-SYN cache
    \item[fbt::tcp\_state\_change:entry] TCP state transition
  \end{description}

  \medskip

  \begin{scriptsize}
    We are using implementation-specific probes (FBT) rather than portable TCP
    probes due to a bug in the FreeBSD/armv7 implementation of DTrace -- the
    last (and most critical!) argument goes missing: the TCP header!  We will
    fix this .. but not today.
  \end{scriptsize}
\end{frame}

\begin{frame}
  \frametitle{Exploratory questions}

  \begin{itemize}
    \item Trace state transitions occurring in test TCP connections
    \item Identify causes of transitions -- packets, system calls (etc)
    \item Varying one-way latency, explore performance of the benchmark with
      TCP
  \end{itemize}
\end{frame}

\begin{frame}
  \frametitle{Experimental questions for the lab report}

  \begin{itemize}
    \item Plot a TCP state-transition diagram for both directions of a flow
    \item Label the state-transition diagram with causes
    \item Compare the diagram with RFC 793
    \item Begin performance analysis of TCP latency vs. throughput
  \end{itemize}

  \medskip

  In the next lab, we will start a causal analysis of why latency affects
  bandwidth in the way that it does
\end{frame}

\begin{frame}
  \frametitle{This lab session}

  \begin{itemize}
    \item Set up IPFW, DUMMYNET, and loopback MTU (see notes)
    \item Ask us if you have any questions or need help
    \item Start with the TCP state machine analysis
  \end{itemize}
\end{frame}

\end{document}
